\section{Einleitung}\label{sec:Einleitung} % 
\noindent Die Kniebeuge ist eine der grundlegendsten Bewegungen des menschlichen Körpers und spielt sowohl im Alltag als auch im Sport eine wichtige Rolle. \cite{Straub2024SquatReview,Meinart} Sie zählt zu den effektivsten Kraftübungen und kann in verschiedenen Wiederholungsbereichen sowie mit unterschiedlichen Variationen ausgeführt werden. Sie bietet zahlreiche Vorteile in den Bereichen Kraftaufbau, Muskelwachstum, Fettverbrennung, Mobilität und Stabilität. \cite{Lindberg2023SquatBenefits}
\\
\noindent Sie ist besonders effektiv für den Muskelaufbau im Unterkörper, insbesondere der Beinmuskulatur. Gleichzeitig stärkt sie gezielt den Rücken und die Rumpfmuskulatur, wodurch sie eine umfassende Ganzkörperübung darstellt. \cite{Schoenfeld2010,Meinart} 

\noindent Falls die klassische Kniebeuge aufgrund bestimmter Pathologien nicht möglich ist, stehen zahlreiche alternative Bewegungsmuster zur Verfügung. Dadurch spielt sie auch in der Rehabilitation eine wichtige Rolle. Zudem trägt sie zur Verbesserung der Beweglichkeit und Stabilität in den Sprunggelenken, Knien und der Hüfte bei, was nicht nur für Sportler, sondern auch für die allgemeine Gesundheit von Bedeutung ist. \cite{Hartmann2014,Lindberg2023SquatBenefits} 

\noindent Die Kniebeuge fördert zudem die Sprungkraft, Explosivität und Sprintleistung, weshalb sie in vielen Sportarten eine zentrale Rolle einnimmt. Aufgrund der hohen Muskelbeanspruchung regt sie zusätzlich den Stoffwechsel an und unterstützt den Fettabbau insbesondere bei höheren Wiederholungszahlen. \cite{Lindberg2023SquatBenefits}

\noindent Insgesamt ist die Kniebeuge eine vielseitige und äußerst wirkungsvolle Übung mit positiven Effekten auf Kraft, Mobilität und sportliche Leistungsfähigkeit. \cite{Hartmann2014,Lindberg2023SquatBenefits,Meinart} 
\\
\\
\noindent Die Kniebeuge zählt zu den grundlegenden funktionellen Bewegungen in Sport und Therapie. Doch welche Kriterien definieren eine qualitativ hochwertige Kniebeuge? Ab welchem Kniewinkel kann von einer vollständigen Ausführung gesprochen werden, und welche Methoden ermöglichen eine objektive Bewertung der Bewegungsausführung?
\noindent Der vorliegende Bericht befasst sich mit der biomechanischen Relevanz der Kniebeuge, erläutert die Anforderungen an eine korrekte Technik und stellt Ansätze zur Erfassung und Bewertung der Bewegungstiefe vor.